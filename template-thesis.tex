\documentclass[a4paper,12pt,twoside,openright]{book}

\usepackage[english]{babel}
\usepackage[T1]{fontenc}
\usepackage[utf8]{inputenc}
\usepackage{lmodern}
\usepackage{fancyhdr}
\usepackage{float}
\usepackage{graphicx}
\usepackage{wrapfig}
\usepackage{setspace}

%------------------------------ colors
\usepackage[usenames,dvipsnames,table]{xcolor} % use colors on table and more
\definecolor{333}{RGB}{51, 51, 51} % define custom color
%------------------------------ source code
\usepackage{listings}
\lstset{
  basicstyle=\footnotesize\sffamily,
  commentstyle=\itshape\color{gray},
  captionpos=b,
  frame=shadowbox,
  language=HTML,
  rulesepcolor=\color{333},
  tabsize=2
}
%------------------------------ define Abstract environment, missing in the 'book' class
\newenvironment{abstract}{\cleardoublepage \null \vfill \begin{center}\bfseries\abstractname \end{center}}{\vfill\null}
%------------------------------ active url
\usepackage{url}
\renewcommand{\UrlFont}{\color{black}\small\ttfamily}
\usepackage[colorlinks=true, linkcolor=black, citecolor=black, urlcolor=black]{hyperref} % active ref
%------------------------------ macros
\newcommand{\sectionname}{Section} % define Section ref
\newcommand{\subsectionname}{Sub-section} % define Sub-section ref
\renewcommand*\arraystretch{1.4} % tables padding


\begin{document}
\frontmatter
\begin{titlepage}
	%immagini di intestazione
	\begin{flushleft}
		\includegraphics[width=0.3\columnwidth]{images/logo_unipd.png}
		\hspace{\fill}
		\includegraphics[width=0.05\columnwidth]{images/logo_DEI.png}
		\hspace{0.01 cm}
		\begin{minipage}[b][1,8 cm][c]{0.3\columnwidth}
			\textsf{{\color{Sepia}{DIPARTIMENTO\\DI INGEGNERIA\\DELL'INFORMAZIONE}}}
		\end{minipage}
	\end{flushleft}
	
	%titoli vari
	\vfill
	\begin{center}
		\begin{large}
			DEPARTMENT OF INFORMATION ENGINEERING
			\\~\\
			BACHELOR'S DEGREE IN COMPUTER ENGINEERING
			\\~\\~\\
			Fusing vision and inertial measurements for autonomous navigation in narrow channels
			\vfill
			Supervisor: 
			\hfill
			Candidate:
			\\
			Prof. Damiano Varagnolo
			\hfill
			Thomas Sanavia
			\vfill
			ACADEMIC YEAR: 2024/2025
			\\
			Graduation date: -- -- ----
			\vfill
		\end{large}
	\end{center}
	
\end{titlepage}


\cleardoublepage % make left page blank
\thispagestyle{empty} %------------------------------ DEDICA

\null
\vspace{2cm}
\begin{flushright}
A ...
\end{flushright}
\vfill

\begin{quote}
  Quote

  \textit{Author}
\end{quote}
\vfill
\null


\begingroup %------------------------------ CONTENTS
  \makeatletter
  \let\ps@plain\ps@empty
  \makeatother
  \tableofcontents
\endgroup

\begin{abstract} %------------------------------ ABSTRACT
\markboth{}{} % remove header
\thispagestyle{empty}
This thesis addresses the design, development, and validation of an automatic system for detecting roll, pitch, and yaw of an autonomous boat. The main objective is to create a low-cost, low-power, and easily transportable solution capable of accurately determining the boat’s attitude. To achieve this, both computer vision and an onboard inertial sensor will be used. By fusing data from these two sources, a precise estimate of the boat’s attitude will be made available to the entire system. For the computer vision part, two cameras will be used to exploit stereo vision. The attitude estimation will be processed on a Raspberry Pi 5, chosen for its low cost. For this purpose, ArUco Tags—two-dimensional fiducial markers commonly used in robotics and augmented reality for pose estimation—will be employed. Regarding the inertial sensor, the onboard device will provide velocities and accelerations along the three axes. Finally, an extended Kalman filter, which will utilize both inertial and computer vision data, will be applied to reduce drift and improve the accuracy of roll, pitch, and yaw estimation.
\end{abstract}



\mainmatter\doublespace 

\chapter*{Introduction} %------------------------------ INTRODUCTION
\addcontentsline{toc}{chapter}{Introduction}
\thispagestyle{empty}
In recent years, the field of \textbf{autonomous systems} has played an increasingly central role in robotics. All this has also been made possible thanks to solutions such as \textbf{Robot Operating System 2 (ROS2)}, which is one of the platforms for the development of robotic applications. This has been achieved thanks to its modular architecture, support for real-time communication, and its open-source nature, which has made it an excellent choice even for advanced robotic systems.  

This thesis is part of the \textbf{Autodocking project}, focused on the ability of a boat to navigate and dock autonomously. Although the structure and the basic control systems were already in place, a computer vision–based system for autonomous navigation was missing. My contribution was related to the creation of a ROS2 node; more specifically, I dealt with the entire pipeline ranging from image acquisition, attitude estimation, and subsequent sensor fusion with the inertial sensor data. To achieve this, I needed a ROS2 node that handled the acquisition of images from the various cameras installed on the boat and their publication on a topic, so that my algorithm could access them.

% --- Pagina bianca con numero romano
%\cleardoublepage
%\null
%\thispagestyle{plain} % mantiene il numero di pagina
%\addtocounter{page}{1} % opzionale, LaTeX normalmente incrementa automaticamente

\chapter{Introduction}
\thispagestyle{empty}
\section{Autonomous Surface Vehicles: challenges and applications}
Unmanned Surface Vehicles (\textbf{USVs}) are particular surface vessels capable of navigating without the presence of a human crew to maneuver them. They can be \textit{remotely operated} through radio or satellite communication, or be completely \textit{autonomous} and controlled by a computer or artificial intelligence; in this case, they are referred to as \textbf{Autonomous Surface Vehicles (ASVs)}. ASVs aim to make journeys \textit{safer}, \textit{reduce costs}, \textit{increase operational continuity}, and carry out \textit{prolonged missions} in \textit{complex} and even \textit{risky} situations for humans.\\

All this has been made possible thanks to significant technological evolution in \textbf{sensors} (satellite, IMU, cameras, LIDAR, radar, etc.), major developments in \textbf{robotics libraries}, and onboard \textbf{computational power} that is increasingly greater and more affordable. Despite this, the \textbf{marine environment} presents major challenges. For example, the \textit{plane of the sea surface} is not a fixed reference but changes independently of the boat’s movements; the surface creates \textit{reflections} that can interfere with instrumentation; and \textit{wind and waves} can put stability control systems to the test. In the marine context, therefore, \textbf{attitude estimation} (roll, pitch, and yaw) is essential, as it is decisive both for the \textit{success of the journey} and for the \textit{precision required in maneuvers}, such as docking.\\

The main challenges concern \textbf{sensors and algorithms}. The first challenge involves \textbf{navigation with GNSS} (Global Navigation Satellite System), which degrades in proximity to port infrastructures, bridges, or areas with many vertical obstacles, as part of the visible satellites are lost. This problem has been partly mitigated with \textbf{inertial sensors (IMU)}, which, however, suffer from another issue: \textit{drift}. The second challenge is the \textbf{detection of possible obstacles} at sea, such as buoys, debris, or even other vessels, often under conditions of \textit{reduced visibility} or \textit{reflections}. The third challenge is \textbf{control and stability}, made complex by \textit{wave motion} that often causes very rapid variations. The last challenge is the ability to make \textbf{real-time decisions}, including very quick changes dictated by sudden shifts in the operational situation.\\

Despite the criticality of these challenges, ASVs are used in various \textbf{applications}. In the \textbf{industrial field}, they are employed for \textit{infrastructure inspection} or \textit{short-range logistical support}. In \textbf{defense and security}, they are used in \textit{special operations} such as \textit{explosive ordnance disposal}. In the \textbf{environmental field}, they are used for \textit{monitoring environmental parameters} and \textit{collecting meteorological data} over extended time horizons, thereby reducing risks for crews.\\

Within this framework, the present thesis focuses on one main issue: obtaining an \textbf{accurate real-time estimation} of the boat’s attitude. To achieve this, \textbf{computer vision techniques} (stereo vision and ArUco markers), \textbf{inertial measurements} through the IMU, and an \textbf{Extended Kalman Filter} will be used to fuse the data, with the aim of \textit{improving the reliability of the estimation}.

\section{Attitude Estimation: roll, pitch, yaw}

To describe the \textbf{attitude} of a vessel, the orientation of the rigid body with respect to a reference system is used. This is done through the three \textbf{Euler angles}\cite{Euler_angles}: \textit{roll} (rotation on the longitudinal axis), \textit{pitch} (rotation on the transverse axis), and \textit{yaw} (rotation on the vertical axis). These three quantities are \textbf{crucial} as they influence \textbf{stabilization} and \textbf{trajectory control} and are fundamental in \textbf{precision maneuvers}.  

\begin{figure}[ht]
  \centering
  \includegraphics[height=6cm]{images/euler_angles.png}
  \caption{Three rotational degrees of freedom of a ASV}\label{unipd-logo}
\end{figure}


\textbf{Attitude estimation} can be obtained from different sources, each with its own advantages and disadvantages. \textbf{Inertial sensors (IMU)}\cite{IMU_Euler}, through gyroscopes and accelerometers, provide roll and pitch with a \textbf{high update frequency}. However, these estimates are affected by \textit{drift}, caused by the accumulation of errors over time. For yaw, a \textbf{magnetometer} or an \textbf{external observation} is required, as the IMU does not provide a direct measurement of this angle. A \textbf{visual sensor} allows estimating the attitude through \textit{known structures} (e.g., ArUco tags) or through the environment if it allows it (e.g., using the \textit{horizon line}). This latter method is less affected by the drift problem but is highly sensitive to \textbf{environmental conditions} and also has a \textbf{lower update frequency}.  

In the \textbf{marine context}, a robust pipeline is represented by the IMU, which provides a \textbf{high-frequency prediction} and, periodically, through \textbf{computer vision}, a measurement relative to the ArUco marker to reduce error accumulation. Through an \textbf{EKF filter}\cite{EKS_IMU_cv}, the different contributions are integrated in order to obtain a \textbf{coherent estimate} even if the conditions are adverse or if one of the sources fails.  

In summary, \textbf{attitude estimation} for an \textbf{ASV} requires a compromise between the \textbf{speed of the readings} and their \textbf{accuracy}. The \textbf{integration} between \textit{inertial} and \textit{visual} measurements provides a \textbf{reliable estimate} capable of supporting \textbf{autonomous navigation}. Subsequently, the different \textbf{implementation choices} to obtain the attitude estimation of the \textbf{Blue Boat} will be presented.  

\section{Computer Vision for Localization and Attitude Estimation}

Computer vision is an essential source for estimating the attitude of the ASV. 
It provides \textbf{roll}, \textbf{pitch}, and \textbf{yaw} with respect to a known reference, 
namely the \textit{ArUco marker}. In our case, it plays a \textit{complementary role} with respect to the IMU, 
which provides an absolute measurement that is not obtained through temporal integration, 
but is instead sensitive to visual conditions (\textit{reflections, backlighting, fog, etc.}).\\
The goal of this chapter is to outline the principles and choices that allow for reliable 
attitude estimation using a \textbf{stereo camera pair}, leveraging \textit{ArUco markers} 
and a geometric calibration of the two cameras.\\

The pipeline is structured into four main steps:
\begin{itemize}
    \item \textbf{Stereo camera calibration:} estimation of intrinsic parameters and distortions, 
    followed by the calculation of rotation and translation between the two cameras.
    \item \textbf{Marker detection:} ArUco markers enable highly reliable identification, 
    provided they are properly scaled according to the distance and have a sufficient pixel resolution.
    \item \textbf{Attitude estimation:} application of a pose estimation algorithm which, 
    knowing the 2D corners and the 3D points of the pattern, can estimate the values of roll, pitch, and yaw.
    \item \textbf{Measurement validation:} a decision filter determines whether to accept the measurement, 
    discard it, or reduce its influence.
\end{itemize}

Triangulation, made more reliable by \textbf{epipolar rectification}, allows for the estimation of depth around the marker. 
These depth cues are used as additional support to improve the attitude estimation.\\

Finally, temporal and synchronization aspects also play a \textbf{crucial role}. 
Synchronization is \textit{decisive} to avoid disparities and inconsistent triangulations. 
The frame rate and the subsequent update time strongly influence the \textbf{accuracy} of the algorithm’s estimation, 
since excessively low values reduce its quality.\\

This approach, entirely based on stereo vision and ArUco markers, enables precise attitude estimation. 
Its accuracy is determined by the quality of calibrations and the robustness of estimation algorithms, 
but it reaches its maximum effectiveness when \textbf{integrated with IMU data}, 
compensating for the weaknesses of each source.

\section{Fiducial markers and ArUco tags}

The \textbf{ArUco marker} is a square planar pattern with a high-contrast black border and an internal matrix encoding a \textit{unique identifier}. This structure allows for fast localization and recognition of the four corners, providing a stable 2D--3D correspondence for attitude estimation. The use of a single marker reduces environmental and economic requirements. However, relying on just one marker requires greater care in localization, calibration, and validation of the measurements.\\

The recognition pipeline adopts a pair of calibrated cameras. After calibration of the individual cameras and subsequent stereo calibration (\textit{rotation, translation, and epipolar rectification}), the images are acquired and rectified, thus aligning the epipolar lines. The detection of the marker involves several steps:

\begin{enumerate}
    \item Adaptive thresholding
    \item Contour extraction
    \item Quadrilateral selection
    \item ID decoding\\
\end{enumerate}

Knowing the marker size, the attitude is calculated through the \textbf{PnP algorithm (Perspective-n-Point)}, obtaining the rotation and translation of the reference camera frame. If both cameras observe the marker, the two estimates can be fused, or the one with higher quality can be selected.\\

The use of a system based on \textbf{stereo vision} improves disparity estimation in the regions close to the marker and enables \textit{metric triangulation} of depth, enhancing stability compared to a monocular approach. This becomes particularly evident when the marker is very small (i.e., far away) or when it is not perpendicular to the viewing angle, as both conditions reduce perspective information.\\

For each estimate, quality indicators are evaluated: reprojection error, effective presence of the marker (all four corners detected), and the angle between the camera and the marker, penalizing views that are too oblique. Furthermore, a comparison is made between the apparent size of the marker in pixels, in order to identify discrepancies due to triangulation or temporal misalignment. \\

From a practical point of view, the use of a single ArUco marker offers a \textbf{simple and easily reproducible implementation path}: a single print on a rigid surface with a matte finish (to reduce reflections) is sufficient. Under these conditions, a single ArUco marker used in a well-calibrated stereo vision pipeline provides roll, pitch, and yaw measurements while maintaining a \textit{good trade-off between precision and computational demand}.
\chapter{Attitude Estimation with Computer Vision}
\thispagestyle{empty}
\section{Vision Hardware and Experimental Setup}
\begin{table}[h]
    \centering
    \begin{tabular}{|c|c|c|c|}
        \hline
        \textbf{Component} & \textbf{Model/Version} & \textbf{Role} & \textbf{Cost} \\
        \hline
        Raspberry Pi 5 & 8GB RAM & Frame acquisition and attitude computation & 85€ \\
        \hline
        Right Pi Camera & V3 & Capture right frame & 27€ \\
        \hline
        Left Pi Camera & V3 & Capture left frame & 27€ \\
        \hline
        Heatsink & Active & CPU cooling & 11€ \\
        \hline
        SanDisk Extreme & 64GB & OS storage & 15€ \\
        \hline
        \textbf{Total} & & & \textbf{165€} \\
        \hline
    \end{tabular}
    \caption{Components used for attitude estimation}
    \label{tab:example}
\end{table}

The computer vision part consists of a pair of rigidly mounted and calibrated cameras. The baseline is 12 cm, also measured during calibration. The cameras operate at a resolution of 1280$\times$720 px with a frame rate of 30 fps, providing an excellent compromise between visual quality and required resources.

The calibration procedure is performed live by framing the calibration checkerboard. Calibrations are carried out for each camera (focal length, principal point, and distortion) and subsequently for the stereo pair (rotation and translation). At the end of the calibration, the files are saved and later loaded by the attitude estimation program.

The visual reference for estimation is a single \textbf{ArUco marker} (in our case, a 6$\times$6 dictionary) with a known size. The marker is printed on a rigid and opaque support to reduce potential problems caused by reflections. The marker size must be such that it remains visible even at a distance, since the acquisition resolution is limited.

Each cycle for the estimation calculation begins with the rectification of the cameras using the previously acquired calibration files. Subsequently, if the marker is detected, the algorithm proceeds to the next step. If present, the four vertices are triangulated to reconstruct the 3D coordinates. The reconstructed points are compared with the theoretical model of the marker, thus allowing the estimation of the \textbf{complete pose} (rotation and translation) using the \textit{estimateAffine3D} method. The rotation is converted to quaternions to avoid 180° jumps in the measurements. Finally, the video stream is displayed with the real-time attitude estimation, showing roll, pitch, and yaw on the screen.

The measurement quality is ensured by checking, in each frame, that all marker corners are visible for correct identification. In this case, the combination of stereo vision, ArUco marker, and pose estimation using the \textit{estimateAffine3D} method provides stable and reliable attitude estimates.


\begin{figure}[ht]
  \centering
  \includegraphics[height=6cm]{images/unipd-light}
  \caption{Image caption}\label{unipd-logo}
\end{figure}

\subsection{Sub-section title}
\begin{wrapfigure}{r}{3cm}
  \vspace{-20pt}
  \begin{center}
  \includegraphics[width=2cm]{images/unipd-bn}
  \end{center}
  \vspace{-10pt}
\end{wrapfigure}

Pellentesque habitant morbi tristique senectus et netus et malesuada fames ac turpis egestas (\seename\ \lstlistingname~\ref{listing01}). Suspendisse arcu magna, faucibus ut tincidunt non, ultrices ut turpis. Nullam tristique vehicula massa, id commodo orci sollicitudin vel. Donec nibh ante, ultrices non facilisis sed, mattis id ligula. Sed sed orci sit amet nulla egestas gravida. Suspendisse laoreet, massa vel sagittis gravida, lectus ligula feugiat risus, a aliquam dolor eros ac orci. Nulla egestas tortor quis nunc scelerisque sed tincidunt massa scelerisque. Pellentesque vulputate pharetra lectus, vitae ultricies nisi luctus eu. Nam congue dui eu quam euismod vitae fermentum sem vehicula. Etiam ac leo id nisi placerat posuere. Curabitur mattis augue eget dolor tempus accumsan consequat diam imperdiet. Sed tristique orci id lacus vulputate rhoncus. Morbi tincidunt ante sed turpis luctus tincidunt et sit amet augue. Cum sociis natoque penatibus et magnis dis parturient montes, nascetur ridiculus mus. Vestibulum ante ipsum primis in faucibus orci luctus et ultrices posuere cubilia Curae; Nunc viverra urna non libero sodales euismod et eleifend sapien. Donec aliquet risus non massa dignissim sollicitudin. Integer a ligula eros. Morbi et lacinia augue~\cite{bookname}.

\begin{lstlisting}[caption={caption text},label=listing01]
<p>
Pellentesque ac tortor eget eros iaculis euismod
vitae vitae augue.
</p>
<!-- comment -->
\end{lstlisting}

Inserimento bibliografico  \cite{IMU_Euler} \cite{Euler_angles} \cite{EKS_IMU_cv}

 

\backmatter

\begingroup %------------------------------ BIBLIOGRAPHY
  \makeatletter
  \let\ps@plain\ps@empty
  \makeatother
  \bibliography{template-thesis}
  \addcontentsline{toc}{chapter}{Bibliografia}
  \bibliographystyle{ieeetr} % sort in order of appearance
\endgroup
\end{document} 