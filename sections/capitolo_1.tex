\chapter{Introduzione} %------------------------------ CHAPTER TITLE
\thispagestyle{empty}
\section{Autonomous Surface Vehicles: challenges and applications}
Unmanned Surface Vehicles (\textbf{USVs}) are particular surface vessels capable of navigating without the presence of a human crew to maneuver them. They can be \textit{remotely operated} through radio or satellite communication, or be completely \textit{autonomous} and controlled by a computer or artificial intelligence; in this case, they are referred to as \textbf{Autonomous Surface Vehicles (ASVs)}. ASVs aim to make journeys \textit{safer}, \textit{reduce costs}, \textit{increase operational continuity}, and carry out \textit{prolonged missions} in \textit{complex} and even \textit{risky} situations for humans.\\

All this has been made possible thanks to significant technological evolution in \textbf{sensors} (satellite, IMU, cameras, LIDAR, radar, etc.), major developments in \textbf{robotics libraries}, and onboard \textbf{computational power} that is increasingly greater and more affordable. Despite this, the \textbf{marine environment} presents major challenges. For example, the \textit{plane of the sea surface} is not a fixed reference but changes independently of the boat’s movements; the surface creates \textit{reflections} that can interfere with instrumentation; and \textit{wind and waves} can put stability control systems to the test. In the marine context, therefore, \textbf{attitude estimation} (roll, pitch, and yaw) is essential, as it is decisive both for the \textit{success of the journey} and for the \textit{precision required in maneuvers}, such as docking.\\

The main challenges concern \textbf{sensors and algorithms}. The first challenge involves \textbf{navigation with GNSS} (Global Navigation Satellite System), which degrades in proximity to port infrastructures, bridges, or areas with many vertical obstacles, as part of the visible satellites are lost. This problem has been partly mitigated with \textbf{inertial sensors (IMU)}, which, however, suffer from another issue: \textit{drift}. The second challenge is the \textbf{detection of possible obstacles} at sea, such as buoys, debris, or even other vessels, often under conditions of \textit{reduced visibility} or \textit{reflections}. The third challenge is \textbf{control and stability}, made complex by \textit{wave motion} that often causes very rapid variations. The last challenge is the ability to make \textbf{real-time decisions}, including very quick changes dictated by sudden shifts in the operational situation.\\

Despite the criticality of these challenges, ASVs are used in various \textbf{applications}. In the \textbf{industrial field}, they are employed for \textit{infrastructure inspection} or \textit{short-range logistical support}. In \textbf{defense and security}, they are used in \textit{special operations} such as \textit{explosive ordnance disposal}. In the \textbf{environmental field}, they are used for \textit{monitoring environmental parameters} and \textit{collecting meteorological data} over extended time horizons, thereby reducing risks for crews.\\

Within this framework, the present thesis focuses on one main issue: obtaining an \textbf{accurate real-time estimation} of the boat’s attitude. To achieve this, \textbf{computer vision techniques} (stereo vision and ArUco markers), \textbf{inertial measurements} through the IMU, and an \textbf{Extended Kalman Filter} will be used to fuse the data, with the aim of \textit{improving the reliability of the estimation}.
Vuoi che ti prepari anche una versione con colori (ad esempio blu per concetti tecnici e